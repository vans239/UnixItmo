\documentclass[14pt]{article} % use larger type; default would be 10pt
\usepackage[utf8]{inputenc} % set input encoding (not needed with XeLaTeX)
\usepackage[english,russian]{babel}
\usepackage{geometry} % to change the page dimensions
\geometry{a4paper} % or letterpaper (US) or a5paper or....
\usepackage{graphicx} % support the \includegraphics command and options
\usepackage{amssymb,amsfonts,amsmath,mathtext,cite,enumerate,float} 
\usepackage{indentfirst}

\AtBeginDocument{\renewcommand{\thesection}{}}
\AtBeginDocument{\renewcommand{\thesubsection}{}}

\title{Task 1}
\author{Ванслов Евгений}
\date{}

\begin{document}
\maketitle
\section{Поиск центра описанной окружности}
\begin{itemize}
\item $O$ - центр
\item $H_a$ -высота из точки A на BC
\item $h_a$ -высота из точки O на BC
\item $\alpha = \angle BAC$
\end{itemize}
$$\begin{cases} 
	R=\frac{abc}{4S}=\frac{a}{2e\sin \alpha} \\
	S=\frac{aH_a}{2}=\frac{bc\sin\alpha}{2} \\
	\sin\alpha=\frac{|[\vec{b},\vec{c}]|}{|\vec{b}||\vec{c}|} \\
\end{cases}$$

$$R^2=h_a^2+(\frac{a}{2})^2$$
$$\begin{cases} 
	h_a=\frac{a}{2}\ctg \alpha \\
	H_a=\frac{bc\sin\alpha}{a} \\
\end{cases}
\Rightarrow\frac{h_a}{H_a}=\frac{a^2\cos\alpha}{2bc\sin^2\alpha}$$
$$O(\frac{a^2\cos\alpha}{2bc\sin^2\alpha},
	\frac{b^2\cos\beta}{2ac\sin^2\beta},
	\frac{c^2\cos\gamma}{2ab\sin^2\gamma})$$

\section{Проверка попадения точки в описанный круг}
Уравнение плоскости, проходящей через 3 заданные точки.
$$ A = 
\begin{pmatrix}
x & y & z & 1 \\
x_1 & y_1 & z_1 & 1 \\
x_2 & y_2 & z_2 & 1 \\
x_3 & y_3 & z_3 & 1 \\
\end{pmatrix}
$$
Если $det|A| <> 0$,то точка находится ниже/выше плоскости в пространстве. Тогда по теореме, доказанной на паре, для окружности будут выполняться те же условия.
$$(x,y) <-> (x,y, x^2 + y^2)$$
$$
\begin{pmatrix}
x & y & x^2 + y^2 & 1 \\
x_1 & y_1 & x_1^2 + y_1^2 & 1 \\
x_2 & y_2 & x_2^2 + y_2^2 & 1 \\
x_3 & y_3 & x_3^2 + y_3^2 & 1 \\
\end{pmatrix}
$$
\end{document}
