\documentclass[14pt]{article} % use larger type; default would be 10pt
\usepackage[utf8]{inputenc} % set input encoding (not needed with XeLaTeX)
\usepackage[english,russian]{babel}
\usepackage{geometry} % to change the page dimensions
\geometry{a4paper} % or letterpaper (US) or a5paper or....
\usepackage{graphicx} % support the \includegraphics command and options
\usepackage{amssymb,amsfonts,amsmath,mathtext,cite,enumerate,float} 
\usepackage{indentfirst}

\AtBeginDocument{\renewcommand{\thesection}{}}
\AtBeginDocument{\renewcommand{\thesubsection}{}}

\title{Task 1}
\author{Ванслов Евгений}
\date{}

\begin{document}
\maketitle
\section{Поиск центра описанной окружности}

Мы знаем, что: $R=\frac{abc}{4S}$.
Также мы знаем формулу герона, поэтому можем посчитать $R$.
Тогда центром описанной окружности является точка пересечения окружностей с радиусом $R$ из двух вершин (пусть $A,B$) такая, что $orient\triangle ABC= orient\triangle ABO$.  

Второе решение лежит на википедии в разделе "Положение центра описанной окружности", его переписывать не хотелось)
Суть решения та же, что было и на паре. (Строим высоту и сравниваем ее длину с длиной серединного перпендикуляра)

\section{Проверка попадения точки в описанный круг}
По первому заданию находим описанную окружность.
Дальше делаем проверку попадения точки в круг.

\begin{itemize}
\itemСтроим любую прямую, пересекающую окружность в двух точках и проходящую через точку ( например, через центр окружности и точку) 
\itemНаходим точки пересечения прямой с окружностью ( $M,N$)
\item $return  (x_M \ge x_O \ge x_N   \&\&  y_M \ge y_O \ge y_N) || (x_M \le x_O \le x_N  \&\&  y_M \le y_O \le y_N) $
\end{itemize}

  

\end{document}
